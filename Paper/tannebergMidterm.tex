\documentclass{scrartcl}\usepackage[]{graphicx}\usepackage[]{color}
%% maxwidth is the original width if it is less than linewidth
%% otherwise use linewidth (to make sure the graphics do not exceed the margin)
\makeatletter
\def\maxwidth{ %
  \ifdim\Gin@nat@width>\linewidth
    \linewidth
  \else
    \Gin@nat@width
  \fi
}
\makeatother

\definecolor{fgcolor}{rgb}{0.345, 0.345, 0.345}
\newcommand{\hlnum}[1]{\textcolor[rgb]{0.686,0.059,0.569}{#1}}%
\newcommand{\hlstr}[1]{\textcolor[rgb]{0.192,0.494,0.8}{#1}}%
\newcommand{\hlcom}[1]{\textcolor[rgb]{0.678,0.584,0.686}{\textit{#1}}}%
\newcommand{\hlopt}[1]{\textcolor[rgb]{0,0,0}{#1}}%
\newcommand{\hlstd}[1]{\textcolor[rgb]{0.345,0.345,0.345}{#1}}%
\newcommand{\hlkwa}[1]{\textcolor[rgb]{0.161,0.373,0.58}{\textbf{#1}}}%
\newcommand{\hlkwb}[1]{\textcolor[rgb]{0.69,0.353,0.396}{#1}}%
\newcommand{\hlkwc}[1]{\textcolor[rgb]{0.333,0.667,0.333}{#1}}%
\newcommand{\hlkwd}[1]{\textcolor[rgb]{0.737,0.353,0.396}{\textbf{#1}}}%

\usepackage{framed}
\makeatletter
\newenvironment{kframe}{%
 \def\at@end@of@kframe{}%
 \ifinner\ifhmode%
  \def\at@end@of@kframe{\end{minipage}}%
  \begin{minipage}{\columnwidth}%
 \fi\fi%
 \def\FrameCommand##1{\hskip\@totalleftmargin \hskip-\fboxsep
 \colorbox{shadecolor}{##1}\hskip-\fboxsep
     % There is no \\@totalrightmargin, so:
     \hskip-\linewidth \hskip-\@totalleftmargin \hskip\columnwidth}%
 \MakeFramed {\advance\hsize-\width
   \@totalleftmargin\z@ \linewidth\hsize
   \@setminipage}}%
 {\par\unskip\endMakeFramed%
 \at@end@of@kframe}
\makeatother

\definecolor{shadecolor}{rgb}{.97, .97, .97}
\definecolor{messagecolor}{rgb}{0, 0, 0}
\definecolor{warningcolor}{rgb}{1, 0, 1}
\definecolor{errorcolor}{rgb}{1, 0, 0}
\newenvironment{knitrout}{}{} % an empty environment to be redefined in TeX

\usepackage{alltt}
\title{Spring 2015 POLSCI.733 MLE Midterm}
\author{Dag Tanneberg}
\date{}

\usepackage[
  pdftex, 
  pdfpagelabels=false, 
  bookmarksopenlevel=section
]{hyperref}
\hypersetup{
  pdftitle = {Midterm Examination Tanneberg}
  pdfauthor = {Dag Tanneberg},
  bookmarksnumbered = true,
  bookmarksopen = false,
  colorlinks = true,
  linkcolor = blue,
  citecolor = blue,
  urlcolor = blue
}

\usepackage{scrpage2}
\lohead{Dag Tanneberg}
\cohead{MLE Midterm}
\rohead{March 17, 2015}
\cfoot{\bfseries\pagemark}
\pagestyle{scrheadings}

\usepackage{amsmath}
\usepackage{amssymb}
\usepackage{graphicx}
\usepackage{lscape}
\usepackage{afterpage}
\IfFileExists{upquote.sty}{\usepackage{upquote}}{}
\begin{document}


\maketitle

Missing data poses severe problems to regression analysis. 
At the very least it diminishes estimation efficiency, 
because less observations are available for analysis. More 
importantly, however, already a small amount of missing data 
may bias regression results severely if missingness is not 
completely unrelated to the observed data and the parameters to 
be estimated (Missing Completely At Random, MCAR). Using the 
example regression of standardized net income inequality on 
ethnic fractionalization and regime type (Polity 2) this effect
becomes readily apparent from Figure \ref{fig:spaghettiPlots}. 
Each panel compares the estimated slope for either predictor 
under different states of missingness. Solid lines 
denote the partial effect without any missing data, 
shaded lines represent estimated slopes from $100$ Amelia 
imputations. Patterned lines denote the partial effect of either 
predictor under list wise deletion respectively averaged over all
Amelia imputations. Finally, the obligatory seed $6886$ is 
given by a dotted line.

Under full information the estimated coefficient for ethnic 
fractionalization is $2.01$ and statistically significant at the
conventional $.95$ level. Hence, judging from the sample higher 
ethnic heterogeneity tends to be associated with higher inequality. 
The reverse holds for political regime type because the partial 
effect for Polity 2 is $-0.096$ and not statistically 
significant. Although higher levels of democracy tend to concur 
with lower levels of inequality this observation should not be 
generalized from the sample. The pattern of missing data 
systematically biases these results and the bias depends on the 
strategy employed to deal with missingness. 

List wise deletion attenuates the effect of ethnic 
fractionalization and renders it statistically insignificant 
(Est.: $.80$, Std. Error: $.44$). In contrast, the coefficient on
Polity 2 increases in absolute size and becomes statistically 
significant (Est.: $-.16$, Std. Error: $0.05$). Thus, list wise 
deletion reverts the `true' process in the data. Multiple 
imputation ranks as a best practice alternative of list wise 
deletion. However, using the obligatory settings it
actually makes things worse because it confirms the results from 
list wise deletion and increases the bias 
(Ethnic fractionalization: Est.: $0.33$, Std. Error:
$0.45$; Polity 2: Est. $-0.29$, Std. Error: $0.04$). This 
perverse result neither depends on the random seed nor on the 
number of imputations as can be seen from the shaded lines. 
Rather, while list wise deletion ignores any causal process that 
might introduce missingness, multiple imputation fails to tap into
it in this exercise.

In conclusion, given non-MCAR data list wise deletion will 
yield misleading results, but multiple imputation may offer 
little improvement if it is not tailored to fit the process 
that induces missingness.

\afterpage{
\begin{figure}[!htb]
  \centering
  \caption{Demonstrating the effect of missing data}
  \label{fig:spaghettiPlots}
  \begin{minipage}{\textwidth}
    \centering
    \includegraphics[scale =.75]{/home/dag/Dropbox/Buero/Dissertation/2015/duke/classes/mle/midterm/out/spaghettiELF.pdf}
  \end{minipage}\\
  \vfill
  \begin{minipage}{\textwidth}
    \centering
    \includegraphics[scale=.75]{/home/dag/Dropbox/Buero/Dissertation/2015/duke/classes/mle/midterm/out/spaghettiPolity.pdf}
  \end{minipage}
\end{figure}
}
\end{document}
