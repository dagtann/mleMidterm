\documentclass{scrartcl}\usepackage[]{graphicx}\usepackage[]{color}
%% maxwidth is the original width if it is less than linewidth
%% otherwise use linewidth (to make sure the graphics do not exceed the margin)
\makeatletter
\def\maxwidth{ %
  \ifdim\Gin@nat@width>\linewidth
    \linewidth
  \else
    \Gin@nat@width
  \fi
}
\makeatother

\definecolor{fgcolor}{rgb}{0.345, 0.345, 0.345}
\newcommand{\hlnum}[1]{\textcolor[rgb]{0.686,0.059,0.569}{#1}}%
\newcommand{\hlstr}[1]{\textcolor[rgb]{0.192,0.494,0.8}{#1}}%
\newcommand{\hlcom}[1]{\textcolor[rgb]{0.678,0.584,0.686}{\textit{#1}}}%
\newcommand{\hlopt}[1]{\textcolor[rgb]{0,0,0}{#1}}%
\newcommand{\hlstd}[1]{\textcolor[rgb]{0.345,0.345,0.345}{#1}}%
\newcommand{\hlkwa}[1]{\textcolor[rgb]{0.161,0.373,0.58}{\textbf{#1}}}%
\newcommand{\hlkwb}[1]{\textcolor[rgb]{0.69,0.353,0.396}{#1}}%
\newcommand{\hlkwc}[1]{\textcolor[rgb]{0.333,0.667,0.333}{#1}}%
\newcommand{\hlkwd}[1]{\textcolor[rgb]{0.737,0.353,0.396}{\textbf{#1}}}%

\usepackage{framed}
\makeatletter
\newenvironment{kframe}{%
 \def\at@end@of@kframe{}%
 \ifinner\ifhmode%
  \def\at@end@of@kframe{\end{minipage}}%
  \begin{minipage}{\columnwidth}%
 \fi\fi%
 \def\FrameCommand##1{\hskip\@totalleftmargin \hskip-\fboxsep
 \colorbox{shadecolor}{##1}\hskip-\fboxsep
     % There is no \\@totalrightmargin, so:
     \hskip-\linewidth \hskip-\@totalleftmargin \hskip\columnwidth}%
 \MakeFramed {\advance\hsize-\width
   \@totalleftmargin\z@ \linewidth\hsize
   \@setminipage}}%
 {\par\unskip\endMakeFramed%
 \at@end@of@kframe}
\makeatother

\definecolor{shadecolor}{rgb}{.97, .97, .97}
\definecolor{messagecolor}{rgb}{0, 0, 0}
\definecolor{warningcolor}{rgb}{1, 0, 1}
\definecolor{errorcolor}{rgb}{1, 0, 0}
\newenvironment{knitrout}{}{} % an empty environment to be redefined in TeX

\usepackage{alltt}
\title{Spring 2015 POLSCI.733 MLE Midterm}
\author{Dag Tanneberg\thanks{%
    All code from this exam will be
    published on March 18 at
    \url{https://github.com/dagtann/mleMidterm}.%
  }
}
\date{}

\usepackage[
  pdftex, 
  pdfpagelabels=false, 
  bookmarksopenlevel=section
]{hyperref}
\hypersetup{
  pdftitle = {Midterm Examination Tanneberg}
  pdfauthor = {Dag Tanneberg},
  bookmarksnumbered = true,
  bookmarksopen = false,
  colorlinks = true,
  linkcolor = blue,
  citecolor = blue,
  urlcolor = blue
}

\usepackage{scrpage2}
\lohead{Dag Tanneberg}
\cohead{MLE Midterm}
\rohead{March 17, 2015}
\cfoot{\bfseries\pagemark}
\pagestyle{scrheadings}

\usepackage{amsmath}
\usepackage{amssymb}
\usepackage{graphicx}
\usepackage{lscape}
\usepackage{afterpage}
\IfFileExists{upquote.sty}{\usepackage{upquote}}{}
\begin{document} 





\maketitle 
\thispagestyle{scrheadings} 

Missing data generates severe problems for regression
analysis. At the very least it diminishes efficiency because
less data is available for parameter estimation. More
importantly, already small amounts of missing data may bias
regression results if missingness is not completely at
random (MCAR), i.e. unrelated to the observed data and the
parameters to be estimated. Using the example regression of
standardized net income inequality on ethnic
fractionalization and regime type (Polity 2) this effect
becomes readily apparent from Figure
\ref{fig:spaghettiPlots}. Each panel compares the estimated
slope for either predictor under different regimes of
missingness.\footnote{In each panel the alternative
predictor has been held constant at the mean of the complete
data.} Solid lines capture the partial effect without
missing data, shaded lines represent estimated slopes from
$100$ Amelia  imputations. Dashed lines report the partial
effect of either  predictor under list wise deletion
respectively averaged over all Amelia imputations. Finally,
the obligatory seed $6886$ is given by a dotted line.

\afterpage{
  \begin{figure}[t]
    \centering
    \caption{The impact of missing data on regression results}
    \label{fig:spaghettiPlots}
    \includegraphics[scale =.8]{/home/dag/Dropbox/Buero/Dissertation/2015/duke/classes/mle/midterm/out/spaghettiELF.pdf} \\
    \includegraphics[scale=.8]{/home/dag/Dropbox/Buero/Dissertation/2015/duke/classes/mle/midterm/out/spaghettiPolity.pdf}
  \end{figure}
}

Under full information the estimated coefficient for ethnic
fractionalization is $2.01$ and statistically significant at
the conventional $.95$ level. Hence, judging from the sample
higher ethnic heterogeneity tends to be associated with
higher inequality. The reverse holds for Polity 2 because
the partial effect of $-0.096$ is not statistically
significant. Although higher levels of democracy tend to
concur with lower levels of inequality this observation
might not travel beyond the sample. The pattern of missing
data systematically biases these results. Thereby the
amount, but not direction, of bias depends on the management
of missing data.

List wise deletion attenuates the effect of ethnic
fractionalization and renders it statistically insignificant
(Est.: $.80$, Std. Er.: $.44$). In contrast, the
coefficient on Polity 2 increases in absolute size and
becomes statistically significant (Est.: $-.16$, Std. Er.:
$0.05$). Thus, list wise deletion misrepresents the `true'
data generating process dramatically. Multiple imputation
ranks as a best practice alternative to list wise deletion.
However, using the obligatory settings for this examination
things turn from bad to worse. Using seed $6886$ Amelia 
confirms the results from list wise deletion and it 
increases the bias (Ethnic fractionalization: Est.: $0.33$, 
Std. Er.: $0.45$; Polity 2: Est. $-0.29$, Std. Er.: $0.04$).
As can be seen from the shaded lines this perverse result 
depends neither on the random seed nor on the number of 
imputations. Rather, while list wise deletion generally 
abstracts from any causal process introducing missingness, 
multiple imputation may fail to tap into it. It is hence no
guarantee for improved inference.
\end{document}
